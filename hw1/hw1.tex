\documentclass[letterpaper, 11pt]{article}
\usepackage{latexsym}
\usepackage{amssymb}
\usepackage{times}
%\usepackage[in]{fullpage}
\usepackage{amsmath,amsfonts,amsthm}
\usepackage{graphicx}

%\documentclass[11pt]{article}
%\pagestyle{myheadings}
%\usepackage[ruled,nothing]{algorithm}
%\usepackage{algorithmic}
%\usepackage[dvips]{epsfig,graphicx}
%\numberwithin{equation}{section}

\bibliographystyle{plain}

\newenvironment{newalgo}[2]{\begin{algorithm}

\caption{\textsc{#1}}\label{#2}

\begin{algorithmic}[1]}{\end{algorithmic}\end{algorithm}}



\newcommand{\gm}{\gamma}
\newcommand{\wh}{\widehat}
\newcommand{\rep}{representation}
\newcommand{\rv}{random variable}
\newcommand{\la}{\lambda}
\newcommand{\wt}{\widetilde}
\newcommand{\st}{such that}
\newcommand{\slvary}{slowly varying}
\newcommand{\ma}{moving average}
\newcommand{\regvary}{regularly varying}
\newcommand{\asy}{asymptotic}
\newcommand{\ts}{time series}
\newcommand{\id}{infinitely divisible}
\newcommand{\seq}{sequence}
\newcommand{\fidi}{finite dimensional \ds}

\newcommand{\ble}{\begin{lemma}}
\newcommand{\ele}{\end{lemma}}
\newcommand{\bfX}{{\bf X}}
\newcommand{\pro}{probabilit}
\newcommand{\BX}{{\bf X}}
\newcommand{\BY}{{\bf Y}}
\newcommand{\BZ}{{\bf Z}}
\newcommand{\BV}{{\bf V}}
\newcommand{\BW}{{\bf W}}
\newcommand{\reals}{{\mathbb R}}
\newcommand{\bbr}{\reals}

\newcommand{\balpha}{\mbox{\boldmath$\alpha$}}
\newcommand{\bbeta}{\mbox{\boldmath$\beta$}}
\newcommand{\bmu}{\mbox{\boldmath$\mu$}}
\newcommand{\tbmu}{\mbox{\boldmath${\tilde \mu}$}}
\newcommand{\bEta}{\mbox{\boldmath$\eta$}}


\def \br#1{\left \{#1 \right \}}
\def \pr#1{\left (#1 \right)}

\newcommand{\Gm}{\Gamma}
\newcommand{\ep}{\epsilon}


\newtheorem{lemma}{Lemma}[section]
\newtheorem{figur}[lemma]{Figure}
\newtheorem{theorem}[lemma]{Theorem}
\newtheorem{proposition}[lemma]{Proposition}
\newtheorem{definition}[lemma]{Definition}
\newtheorem{corollary}[lemma]{Corollary}
\newtheorem{example}[lemma]{Example}
\newtheorem{exercise}[lemma]{Exercise}
\newtheorem{remark}[lemma]{Remark}
\newtheorem{fig}[lemma]{Figure}
\newtheorem{tab}[lemma]{Table}
\newtheorem{fact}[lemma]{Fact}
\newtheorem{test}{Lemma}
\newtheorem{algorithm}[lemma]{Algorithm}

\newcommand{\play}{\displaystyle}

\newcommand{\ms}{measure}
\newcommand{\beao}{\begin{eqnarray*}}
\newcommand{\eeao}{\end{eqnarray*}\noindent}
\newcommand{\beam}{\begin{eqnarray}}
\newcommand{\eeam}{\end{eqnarray}\noindent}

\newcommand{\halmos}{\hfill\mbox{\qed}\\}
\newcommand{\fct}{function}
\newcommand{\ins}{insurance}
\newcommand{\ds}{distribution}

\newcommand{\one}{{\bf 1}}
\newcommand{\eid}{\buildrel{\rm d}\over {=}}
\newcommand {\Or}{\rm ORDER}
\newcommand {\In}{\rm INTER}

\newcommand{\bbd}{{\mathbb D}}
\newcommand{\vi}{$V_{ij}$ }
\newcommand{\rr}{R^{\prime\prime}}
%\newcommand{\R}{R^\prime}
\newcommand{\ci}{\frac{1}{c}}
\newcommand{\Vi}{V(n)}
\newcommand{\dR}{\mathcal R}
\newcommand{\md}[1]{\left(\ \rm{mod}\ \it{#1}\right)}
\newcommand{\So}{s}
%\begin{document}
%\def\DoubleSpace{\baselineskip=24pt}
%\DoubleSpace \sloppy

\begin{document}



\title{Homework \#1 \\ Introduction to Algorithms/Algorithms 1 \\ 600.363/463 \\ Spring 2013}
\author{\textbf{Due on:} Tuesday, February 5th, 5pm \\
\textbf{Late submissions:} will NOT be accepted\\
\textbf{Format:} Please start each problem on a new page.
\\\textbf{Where to submit:} On blackboard, under student assessment. \\
Otherwise, please bring your solutions to the lecture.
\\}


\maketitle



\section{Problem 1 (20 points)}

\subsection{(10 points)}




For each statement below explain if it is true or false and prove
your answer. Be as precise as you can. The base of $\log$ is $2$ unless stated otherwise.
\begin{enumerate}

\item  $7^{39}(n+343n^2) = \Theta(n^{2})$

\item $2^n = \Theta(e^{n+\sqrt{n}})$

\item $e^n = \Theta(2^{(3n)})$

\item $e^n = \Theta(2^{(n+3)})$


\item $\log_2(n^7) = O(\log_e(n^{10000000000}))$

\item $\arctan(n) = O(n)$

\item Let $f,g$ be positive functions. Then $f(n)+g(n) = \Theta(\min(f(n),g(n)))$

\item $n^{\log\log(n)} = \Omega(\left(\log(n^{100})\right)^{1.000001})$

\item Let $f$ be a positive function. Then $f(n)+o(f(n)) = \Theta(f(n))$.


\end{enumerate}


\subsection{(10 points)}

\begin{enumerate}

\item Prove that $$\sum_{i=\sqrt{n}}^n {1\over i^3} = O({1\over \sqrt{n}}).$$


\end{enumerate}

\section{Problem 2(20 Points)}

\subsection{(10 points)}

Prove by induction that $4
^n+6n +8$ is divisible by $9$ for all $n >= 1$.


\subsection{(10 points)}

\begin{enumerate}

\item Let $A,B,C,D$ be sets. Prove that $$(A\cap B) \cup (C\cap D) = (A\cup C) \cap (B\cup C) \cap (A\cup D) \cap (B\cup D).$$

\item There are $10$ cookies of different colors in the jar. What is the number of different ways to divide the cookies between Alice and Bob? What if the cookies are of the same color (i.e., identical)?

\item We have 50 balls. Each ball, independently and randomly, is placed into one of $5$ bins. What is the probability that there are no empty bins at the end of our experiment?

\item There are $25$ students in the class. Prove that there are always at least three
students whose birthdays are in the same month.
\end{enumerate}


\end{document}



































