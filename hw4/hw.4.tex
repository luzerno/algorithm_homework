\documentclass[letterpaper, 11pt]{article}
\usepackage{latexsym}
\usepackage{amssymb}
\usepackage{times}
%\usepackage[in]{fullpage}
\usepackage{amsmath,amsfonts,amsthm}
\usepackage{graphicx}
\usepackage{color}
\usepackage{algorithmic}
\usepackage{algorithm}
%\documentclass[11pt]{article}
%\pagestyle{myheadings}
%\usepackage[ruled,nothing]{algorithm}
%\usepackage{algorithmic}
%\usepackage[dvips]{epsfig,graphicx}
%\numberwithin{equation}{section}

\bibliographystyle{plain}

\newenvironment{newalgo}[2]{\begin{algorithm}

\caption{\textsc{#1}}\label{#2}

\begin{algorithmic}[1]}{\end{algorithmic}\end{algorithm}}



\newcommand{\wh}{\widehat}
\newcommand{\rep}{representation}
\newcommand{\gm}{\gamma}
\newcommand{\rv}{random variable}
\newcommand{\la}{\lambda}
\newcommand{\wt}{\widetilde}
\newcommand{\st}{such that}
\newcommand{\slvary}{slowly varying}
\newcommand{\ma}{moving average}
\newcommand{\regvary}{regularly varying}
\newcommand{\asy}{asymptotic}
\newcommand{\ts}{time series}
\newcommand{\id}{infinitely divisible}
\newcommand{\seq}{sequence}
\newcommand{\fidi}{finite dimensional \ds}

\newcommand{\ble}{\begin{lemma}}
\newcommand{\ele}{\end{lemma}}
\newcommand{\bfX}{{\bf X}}
\newcommand{\pro}{probabilit}
\newcommand{\BX}{{\bf X}}
\newcommand{\BY}{{\bf Y}}
\newcommand{\BZ}{{\bf Z}}
\newcommand{\BV}{{\bf V}}
\newcommand{\BW}{{\bf W}}
\newcommand{\reals}{{\mathbb R}}
\newcommand{\bbr}{\reals}

\newcommand{\balpha}{\mbox{\boldmath$\alpha$}}
\newcommand{\bbeta}{\mbox{\boldmath$\beta$}}
\newcommand{\bmu}{\mbox{\boldmath$\mu$}}
\newcommand{\tbmu}{\mbox{\boldmath${\tilde \mu}$}}
\newcommand{\bEta}{\mbox{\boldmath$\eta$}}


\def \br#1{\left \{#1 \right \}}
\def \pr#1{\left (#1 \right)}

\newcommand{\Gm}{\Gamma}
\newcommand{\ep}{\epsilon}


\newtheorem{lemma}{Lemma}[section]
\newtheorem{figur}[lemma]{Figure}
\newtheorem{theorem}[lemma]{Theorem}
\newtheorem{proposition}[lemma]{Proposition}
\newtheorem{definition}[lemma]{Definition}
\newtheorem{corollary}[lemma]{Corollary}
\newtheorem{example}[lemma]{Example}
\newtheorem{exercise}[lemma]{Exercise}
\newtheorem{remark}[lemma]{Remark}
\newtheorem{fig}[lemma]{Figure}
\newtheorem{tab}[lemma]{Table}
\newtheorem{fact}[lemma]{Fact}
\newtheorem{test}{Lemma}

\newcommand{\play}{\displaystyle}

\newcommand{\ms}{measure}
\newcommand{\beao}{\begin{eqnarray*}}
\newcommand{\eeao}{\end{eqnarray*}\noindent}
\newcommand{\beam}{\begin{eqnarray}}
\newcommand{\eeam}{\end{eqnarray}\noindent}

\newcommand{\halmos}{\hfill\mbox{\qed}\\}
\newcommand{\fct}{function}
\newcommand{\ins}{insurance}
\newcommand{\ds}{distribution}

\newcommand{\one}{{\bf 1}}
\newcommand{\eid}{\buildrel{\rm d}\over {=}}
\newcommand {\Or}{\rm ORDER}
\newcommand {\In}{\rm INTER}

\newcommand{\bbd}{{\mathbb D}}
\newcommand{\vi}{$V_{ij}$ }
\newcommand{\rr}{R^{\prime\prime}}
%\newcommand{\R}{R^\prime}
\newcommand{\ci}{\frac{1}{c}}
\newcommand{\Vi}{V(n)}
\newcommand{\dR}{\mathcal R}
\newcommand{\md}[1]{\left(\ \rm{mod}\ \it{#1}\right)}
\newcommand{\So}{s}
%\begin{document}
%\def\DoubleSpace{\baselineskip=24pt}
%\DoubleSpace \sloppy
\newcommand{\feedback}[1]{\textcolor{red}{#1}}
\begin{document}



\title{Homework \#4 \\ Introduction to Algorithms/Algorithms 1 \\ 600.363/463 \\Spring 2013}
\author{Yifan Ge}

\maketitle

%%%%%%%%%%%%%%%%%%%%%%%%%%%%%%%%%
\section{Problem 1 (20 points)} % #6
Let $A$ be an array of numbers, where the size of the array is $N$.
The numbers represent winners in different rounds of game $G$.
We say that there is an undisputed champion if one element occupies more than half of entries.
For example, if $A = \{1,2,2,3, 2\}$ then $2$ is the undisputed champion.
Design an algorithm that checks if there is a undisputed champion in $A$.
Full credit will be given if your algorithm works in $O(N)$ time.

Partial credit will be given if your algorithm works in $O(N\log(N))$ time.\\

\textbf{Solution:} We can observe that if $k$ is the undisputed champion of $A$, $k$ must be the median of $A$. So we can use the algorithm described in Chapter II to find the median in $O(n)$ time. Then we need to verify if the median is the undisputed champion. This can be done by looping through $A$ and count the occurrences of median found. If the median occupies more than half of the entries, it is the undisputed champion; if not, there is no undisputed champion in $A$. This step also runs in $O(n)$ time. So the algorithm works in $O(N)$ time.\\
Algorithm \ref{algo:champion} shows the pseudocode for this algorithm.
\begin{algorithm}
\label{algo:champion}
\caption{Check if there is a undisputed champion in array $A$}
\begin{algorithmic}[1]
    \STATE $k \leftarrow FindMedian(A)$ 
    \STATE $cnt \leftarrow 0$
    \FOR{$i \leftarrow 1$ \textbf{to} $length(A)$}
        \IF{$A[i] == k$}
            \STATE $cnt++$
        \ENDIF
    \ENDFOR
    \IF{$cnt \le \lceil \frac{length(A)}{2} \rceil$}
        \RETURN $true$
    \ELSE
        \RETURN $false$
    \ENDIF
\end{algorithmic}
\end{algorithm}

\section{Problem 2 (20 points)}
\subsection{(10 points)}
Let A be an array of size $n$ of integers with the property that the number of distinct integers is  $O(\log(n))$. Design an efficient $O(n\log\log(n))$ algorithm to sort $A$.\\
\textbf{Solution:} This can be done using a self-balancing binary search tree like Red-black tree. 
\\Loop through the array $A$, and insert each element into a Red-black tree. Each node in the tree keeps a counter to record the number of occurrences of the element. As the number of distinct integers is $O(\log(n))$, the height of the Red-black tree will not exceeds $O(\log\log(n))$, so the time complexity of inserting an element into the tree is $O(\log\log(n))$. As there are $n$
elements in the array, the total time complexity of constructing the Red-black tree is $n\log\log(n)$.\\
Now traverse the Red-black tree in in-order to repopulate a sorted array. If the counter on a node is $k$, the element needs to be repeated for $k$ times. Traversing the tree works in $O(\log(n))$ time as there are $O(\log(n))$ nodes in the tree and repopulating the array works in $O(n)$ time as the number of integers is $O(n)$, so the overall time complexity of this step is $O(n)$.\\
As a whole, the overall time complexity of the algorithm is $O(n\log\log(n)) + O(n)=O(n\log\log(n))$.

\subsection{(10 points)}
Let A be an array of size $n$ of integers with the property that the first $n-\sqrt{n}$ integers are sorted. Design an efficient $o(n\log(n))$ algorithm to sort $A$.\\

\textbf{Solution:} First use randomized quicksort to sort the last $\sqrt(n)$ integers, and this can be done in $O(\sqrt{n}\log{\sqrt{n}}$ time. Then use the merge operation in mergesort to merge the first $n-\sqrt{n}$ integers and the last $\sqrt{n}$ integers. This step runs in $n-\sqrt{n}$ time as $n-\sqrt{n}>\sqrt{n}$ when $n$ is sufficiently large. The time complexity of the whole algorithm is $O(\sqrt{n}\log{\sqrt{n}}) +
O(n-\sqrt{n}=O(\sqrt{n}(\frac{1}{2}\log{n}+\sqrt{n}-1)))=O(n)$, which means the algorithm runs in $o(n\log{n})$ time.

$$
\begin{pmatrix}
    \cos(45^{\circ}) & -\sin(45^{\circ}) & 15 \\
    \sin(45^{\circ}) & \cos(45^{\circ}) & 0\\
    0 & 0 & 1
\end{pmatrix}
$$
\end{document}


%%%%%%%%%%%%%%%%%%%%%%%%%%%%%
